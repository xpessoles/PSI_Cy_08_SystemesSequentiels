\documentclass[10pt,fleqn]{article} % Default font size and left-justified equations
\usepackage[%
    pdftitle={Modélisation dynamique},
    pdfauthor={Xavier Pessoles}]{hyperref}

    
\input{style/new_style}
\input{style/macros_SII}
\usepackage{multicol}
\usepackage{siunitx}
%\usepackage{picins}
\fichetrue
%\fichefalse

\proftrue
\proffalse

\tdtrue
%\tdfalse

\courstrue
\coursfalse


\def\classe{\textsf{PSI$\star$ -- MP}}
\def\xxnumpartie{Cycle 08}
\def\xxpartie{Analyse de la chaîne d'information des systèmes}

\def\xxnumchapitre{Chapitre 2 \vspace{.2cm}}
\def\xxchapitre{\hspace{.12cm} Systèmes séquentiels}


\def\discipline{Sciences \\Industrielles de \\ l'Ingénieur}
\def\xxtete{Sciences Industrielles de l'Ingénieur}




\def\xxtitreexo{TD 02 -- Coffre motorisé A6}
\def\xxsourceexo{\hspace{.2cm} M. Libourel}


\def\xxposongletx{2}
\def\xxposonglettext{1.45}
\def\xxposonglety{20}
%\def\xxonglet{Part. 1 -- Ch. 3}
\def\xxonglet{\textsf{Cycle 08}}

\def\xxactivite{TD 02}
\def\xxauteur{\textsl{M. Libourel}\\\textsl{Lycée Clémenceau}}

\def\xxcompetences{%
\textsl{%
\textbf{Savoirs et compétences :}\\
%\begin{itemize}[label=\ding{112},font=\color{ocre}] 
%\item \textit{Mod2.C34} : chaînes de solides.
%\item \textit{Mod2.C34} : degré de mobilité du modèle;
%\item \textit{Mod2.C34} : degré d’hyperstatisme du modèle;
%\item \textit{Mod2.C34.SF1} : déterminer les conditions géométriques associées à l’hyperstatisme;
%\item \textit{Mod2.C34} : résoudre le système associé à la fermeture cinématique et en déduire le degré de mobilité et d’hyperstatisme.
%\end{itemize}
}}

\def\xxfigures{
\includegraphics[width=.6\linewidth]{images/fig_01}
}%figues de la page de garde

\def\xxpied{%
Cycle 08 -- Analyse de la chaîne d'information des systèmes \\%dans le but de déterminer les contraintes géométriques dans les mécanismes\\% afin de valider leurs performances.\\
Chapitre 2 -- \xxactivite%
}

\setcounter{secnumdepth}{5}
%---------------------------------------------------------------------------

\usepackage{pgfplots}
\begin{document}
\def\pathfig{images}
%\chapterimage{png/Fond_Cin}
\input{style/new_pagegarde}
\vspace{4.5cm}
\pagestyle{fancy}
\thispagestyle{plain}

\def\columnseprulecolor{\color{ocre}}
\setlength{\columnseprule}{0.4pt} 

\def\pathfig{images}


\ifprof
\begin{multicols}{2}
\else
\begin{multicols}{2}
\fi

\subsection*{Mise en situation}


Depuis 2005, un coffre motorisé est proposé en option sur l’Audi A6. Ce système développé par la société Valéo été récompensé en 2002 par le prix de l’innovation électronique automobile EPCOS/SIA dans la catégorie « Vie à bord, confort, habitacle ». La motorisation du hayon permet l’ouverture ou la fermeture automatique du coffre. L’ouverture s’effectue soit à l’aide de la télécommande, soit par action sur une touche située à proximité du conducteur, soit par action sur une touche située sur la poignée du hayon. La fermeture s’effectue par action sur une touche située sur la face interne du hayon. L’utilisateur a la possibilité de programmer l’angle d’ouverture du hayon pour éviter par exemple qu’il ne heurte le plafond du garage. L’utilisateur conserve naturellement la possibilité de man\oe{}uvrer manuellement le hayon. Ce système dispose également de détecteurs d’obstacles. En position fermée, le système doit assurer le blocage du hayon avec la caisse du véhicule. Une expression partielle des besoins durant la phase d’utilisation est donnée par le diagramme d’exigences partiel.


\begin{center}
\includegraphics[width=.8\linewidth]{images/fig_02.png}

\includegraphics[width=.9\linewidth]{images/fig_03.png}
\end{center}

\subsection*{Présentation du système}
Présentation du système
Le schéma d’implantation du système est décrit sur la figure ci-dessous. Ce système se compose :
\begin{itemize}
\item de deux unités électromécaniques (une sur chaque face latérale du hayon) permettant de man\oe{}uvrer électriquement le hayon et renseignant les contrôleurs sur la position du hayon et sur la présence éventuelle d’obstacles ;
\item de deux contrôleurs (un par unité électromécanique) pilotant les moteurs électriques des unités en fonction des lois de commande en vitesse. Les deux contrôleurs sont reliés entre eux par un réseau LIN et seul le contrôleur maître communique avec le calculateur Gateway par le bus « confort » ;
\item d’un calculateur Gateway gérant l’ensemble des composants (contrôleur maître, serrure électrique, gâche motorisée) du système de coffre motorisé en fonction des consignes de l’utilisateur et de la situation de la partie opérative ;
\item d’une serrure électrique, solidaire du hayon, permettant en position fermée de verrouiller le hayon avec la gâche ;
\item d’une gâche motorisée, solidaire de la caisse du véhicule, permettant en position fermée de plaquer le hayon contre la caisse en tirant la serrure.
\end{itemize}


\subsection*{Validation partielle de l’exigence id1.1}
\begin{obj}
L’objectif est de vérifier la coordination des activités en fonctionnement normal ainsi que le réglage de l’ouverture maximale du hayon.
\end{obj}

Le contrôleur commande le moteur et l’embrayage de l’unité électromécanique en fonction des informations provenant essentiellement du calculateur Gateway. Le diagramme d’états partiel qui décrit le fonctionnement normal est codé et implémenté dans le contrôleur maître. 

\subsubsection*{Description des entrées du diagramme d’états}

\begin{itemize} 
\item Des capteurs angulaires à effet Hall permettent de mesurer la position angulaire $\gamma$ du hayon.
\item Le calculateur Gateway délivre les informations binaires :
\begin{itemize}
\item $t_o=1$ si une pression est exercée sur l’une des touches d’ouverture automatique ;
\item $t_f=1$ si une pression est exercée sur la touche de fermeture située sur la face interne du hayon ;
\item $p=1$ si l’utilisateur agit directement sur la poignée du hayon.
\end{itemize}
\end{itemize}

\subsubsection*{Description des activités du diagramme d’états}
L’ouverture et la fermeture automatique du hayon sont réalisées par un moteur électrique $M$ à courant continu et à aimants permanents alimenté par un hacheur quatre quadrants [$(M+)$: ouvrir le hayon ; $(M-)$  : fermer le hayon].

La modulation du couple transmissible par l’embrayage s’obtient en modifiant la pression de contact sur la garniture du disque d’embrayage. Cette pression est fonction de l’intensité du champ magnétique résultant d’un aimant permanent et d’un électro-aimant ($E$). En phase d’ouverture automatique, le champ magnétique de l’électroaimant ($E+$) vient s’ajouter à celui de l’aimant permanent, alors qu’en phase de fermeture automatique, l’électroaimant n’est pas alimenté. Dans le cas d’une man\oe{}uvre manuelle du hayon, le moteur est désaccouplé grâce au champ magnétique de l’électroaimant $(E-)$ qui s’oppose à celui de l’aimant permanent. Les positions limites basse et haute du hayon valent respectivement $\gamma=0\degres$ (coffre fermé) et $\gamma=\gamma_{\text{MAXI}}$ (coffre ouvert).
On suppose qu’en mode automatique la vitesse de $\SI{20}{\degres.s^{-1}}$ en ouverture ou en fermeture du hayon est atteinte instantanément.




\subparagraph{}
\textit{Sur le document réponse, compléter le chronogramme d’évolution du diagramme d’états partiel de fonctionnement normal sachant qu’à l’instant initial, le coffre est fermé et que la valeur préprogrammée de  $\gamma_{\text{MAXI}}$ est de $90\degres$. }
\ifprof
\begin{corrige}
\end{corrige}\else\fi

Le chronogramme précédent laisse apparaître que l’utilisateur a modifié la valeur $\gamma_{\text{MAXI}}$.


\subparagraph{}
\textit{Quelle est la nouvelle valeur de $\gamma_{\text{MAXI}}$ ?}
\ifprof
\begin{corrige}
\end{corrige}\else\fi 


\subparagraph{}
\textit{Comment l’utilisateur doit-il procéder afin d’augmenter $\gamma_{\text{MAXI}}$ ?
}
\ifprof
\begin{corrige}
\end{corrige}\else\fi

\ifprof
\end{multicols}
\else
\end{multicols}
\fi

\begin{center}
\includegraphics[width=.75\linewidth]{images/fig_05.png}

\includegraphics[width=.75\linewidth]{images/fig_04.png}
\end{center}


\begin{center}
\includegraphics[width=\linewidth]{images/fig_06.png}
\end{center}

\end{document}





\subparagraph{}
\textit{}
\ifprof
\begin{corrige}
\end{corrige}\else\fi


